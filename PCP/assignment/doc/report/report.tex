\documentclass[a4paper,10pt,openright,openbib,twocolumn]{article}
%\usepackage[portuges]{babel}
\usepackage[T1]{fontenc}
\usepackage{ae}
\usepackage[utf8]{inputenc}
\usepackage[pdftex]{graphicx}
\usepackage{url}
\usepackage{listings}
\usepackage{verbatim}
\usepackage{enumerate}
\usepackage[pdftex, bookmarks, colorlinks, linkcolor=black, urlcolor=blue]{hyperref} 
%\usepackage[a4paper,left=2.5cm,right=2.5cm,top=3.5cm,bottom=3.5cm]{geometry}
\usepackage[paper=a4paper,top=2cm,left=2cm,right=1.5cm,bottom=2cm,foot=1cm]{geometry}
\usepackage{colortbl}
\usepackage[margin=10pt,font=small,labelfont=bf]{caption}
\usepackage{mdwlist}
\usepackage{cleveref}
\usepackage{epsfig}

\usepackage{multicol}
\usepackage{appendix}
\usepackage{listings}



\setlength{\parindent}{0cm}
\setlength{\parskip}{2pt}


\begin{document}
\title{The Cilk Plus Extension}
\date{\today}
\begin{multicols}{2}
\author{
    Brito, Rui\\
    PG22781\\
    Department of Informatics\\
    University of Minho\\
    ruibrito666@gmail.com
  \and
    Alves, José\\
    PG22765\\
    Department of Informatics\\
    University of Minho\\
    zealves.080@gmail.com
}
\date{}
\maketitle
\end{multicols}

\begin{abstract}
    This report presents an analysis of the Cilk Plus extension for the C and C++ langagues. Cilk Plus is not just a library, it is an extension implemented by the compiler and the Cilk Plus runtime, thus allowing for lower overhead than some of it's competitors. We compare Cilk Plus with a very well known alternative, OpenMP, and, although some results were positve (namely, recursive algorithms), overall, OpenMP wins. 
\end{abstract}

\section{Introduction}    %% Reler, corrigir algum erro
    Cilk Plus provides an easy way to exploit parallelism inside an application, by using only three keywords, one can fully take advantage of both multicore and SIMD extensions. 
\end{document}
