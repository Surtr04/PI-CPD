\documentclass[a4paper,10pt,openright,openbib]{article}
\usepackage[portuges]{babel}
\usepackage[T1]{fontenc}
\usepackage{ae}
\usepackage[utf8]{inputenc}
\usepackage[pdftex]{graphicx}
\usepackage{url}
\usepackage{listings}
\usepackage{verbatim}
\usepackage{enumerate}
\usepackage[pdftex, bookmarks, colorlinks, linkcolor=black, urlcolor=blue]{hyperref} 
\usepackage[a4paper,left=2.5cm,right=2.5cm,top=3.5cm,bottom=3.5cm]{geometry}
\usepackage{colortbl}
\usepackage[margin=10pt,font=small,labelfont=bf]{caption}
\usepackage{mdwlist}
\usepackage{cleveref}

\usepackage{amsmath}
\usepackage[retainorgcmds]{IEEEtrantools}


\setlength{\parindent}{0cm}
\setlength{\parskip}{2pt}


\begin{document}

%% Roofline%%
\section{Roofline}
\subsection{Machines Profile}
The machines used for this study were a MacBook Pro late 2008 and a HP dv6-2190ep from 2010.
The information about the machines were gathered from \emph{/proc/cpuinfo}, \emph{/proc/meminfo}, the \emph{Intel Ark} \cite{ark} and \emph{Crucial} \cite{crucial} websites and with \emph{dmidecode} and \emph{sysctl} linux tools.

\subsubsection{Peaks}
In order to calculate the rooflines, we needed the Floating-Point(FP) Performance Peak and the Memory Bandwidth's Peak. 
To attain the FP Performance Peak we calculate the following formula:

$$\mathrm{GFlop/s_{max}} =  \#_{\mathrm{cores}} \times f_{\mathrm{clock}} \times \#_{\mathrm{SIMD}}$$

MacBook Pro FP Performance Peak:
$$\mathrm{GFlop/s_{max}} =  2 \times 2.8 \times 8
						 =  44.8 GFLOPS\\s$$
HP Pavillion FP Performance Peak:
$$\mathrm{GFlop/s_{max}} =  4 \times 1.6 \times 8
						 =  51.2 GFLOPS\\s$$

To calculate de Memory Bandwidth Peak we resolve the following formula:

$$\mathrm{BW_{max}} =  \#_{\mathrm{channels}} \times mem_{\mathrm{clock}} \times bus_{\mathrm{bandwidth}}$$
 
MacBook Pro Memory Bandwidth Peak:
$$\mathrm{GFlop/s_{max}} =  2 \times 1067 \times 64
						 =  17.072 GB\\byte$$
HP Pavillion Memory Bandwidth Peak:
$$\mathrm{GFlop/s_{max}} =  2 \times 1333 \times 64
						 =  21.328 GB\\byte$$

\subsubsection{Specifications}
The specifications of the MacBook Pro are displayed on \autoref{tab:mbp}. \\

\begin{table}[!htp]
		\begin{tabular}{lrl}
			\hline \\
			\textbf{Manufacter:} & Apple \\
			\hline \\
			\textbf{Model:} & MacBook Pro late 2008 \\
			\hline \\
			\textbf{Processor} & & \\
			Manufacturer: & Intel & \\
			Arch: & Core & \\
			Model: & Core 2 Duo T9600 & \\
			Cores: & 2 & \\
			Clock Frequency: & 2.80 GHz & \\
			FP Performance's Peak: & 44.8 GFlops/s & \\
			\hline \\
			\textbf{Cache} & & \\
			Level: & 1 & \\
			Size: & 32KB + 32KB & \\
			Line Size: & 64 B & \\
			Associative: & 8-way & \\
			Memory Access Bandwidth: & 40 GB/s & \\
			\\
			Level: & 2 & \\
			Size: & 6 MB & \\
			Line Size: & 64 B & \\
			Associative: & 24-way & \\
			\hline \\
			\textbf{RAM} \\
			Type: & SDRAM DDR3 PC3-8500 & \\
			Frequency: & 1067 MHz & \\
			Size: & 4 GB & \\
			Num. Channels: & 2 & \\
			Latency: & 13.13 ns & \\
		\end{tabular}
		\caption{MacBook Pro late 2008 specifications}
		\label{tab:mbp}
\end{table}
The specifications of the HP dv6-2190ep are displayed on \autoref{tab:hp}. \\
\begin{table}[!htp]
		\begin{tabular}{lrl}
			\hline \\
			\textbf{Manufacter:} & HP \\
			\hline \\
			\textbf{Model:} & Pavillion dv6-2190ep \\
			\hline \\
			\textbf{Processor} & & \\
			Manufacturer: & Intel & \\
			Arch: & Nehalem & \\
			Model: & i7-720QM & \\
			Cores: & 4 & \\
			Clock Frequency: & 1.60 GHz & \\
			FP Performance's Peak: & 51.2 GFlops/s & \\
			\hline \\
			\textbf{Cache} & & \\
			Level: & 1 & \\
			Size: & 32KB + 32KB & \\
			Line Size: & 64 B & \\
			Associative: & 4/8-way & \\
			Memory Access Bandwidth: & 22 GB/s & \\
			\\
			Level: & 2 & \\
			Size: & 256 KB & \\
			Line Size: & 64 B & \\
			Associative: & 8-way & \\
			\\
			Level: & 3 & \\
			Size: & 6 MB & \\
			Line Size: & 64 B & \\
			Associative: & 12-way & \\
			\hline \\
			\textbf{RAM} \\
			Type: & SDRAM DDR3 PC3-10600 & \\
			Frequency: & 1333 MHz & \\
			Size: & 4GB & \\
			Num. Channels: & 2 & \\
			Latency: & 13.5 ns & \\
		\end{tabular}
		\caption{HP Pavillion dv6-2190ep specifications}
		\label{tab:hp}
\end{table}


\subsection{Roofline Model}
\subsubsection{Machines' Roofline}
IMAGEM MBP

IMAGEM HP

\subsubsection{Ceilings}
As suggested by the Roofline paper we added several ceilings to understand wich optimizations we may perform.
This ceilings were given by recalculating the roofline without some key charateristics.
\begin{description}
\item[Peak floating-point performance] The roofline, where all components and features are considered.
\item[]
\item[]
\item[]
\end{description}
For memory only one ceiling was calculated, besides the roofline.
\begin{description}
\item[Peak stream bandwidth] The roofline, where all features are considered.
\item[One-channel]
\end{description}

\section{PAPI Case Study}
\subsection{Problem}
The case study of this report, is to analyse the performance of a \textbf{matrix multiplication} algorithm, \begin{equation}Matrix A * Matrix B = Matrix C\end{equation} wich contains a triple nested loop with the indexes i,j and k(line,column and position). Our implementation will explore the index order \textbf{i,j,k} of the triple nested loop.

\subsection{Algorithm Analysis}
The algorithm produced to calculate the matrix multiplication is:

\begin{verbatim}
for (i = 0; i < size; i++) {
    for (j = 0; j < size; j++) {
        for(k = 0; k < size; k++) {
            acc += matrixA[i][k] * matrixB[k][j];				
            }		
            matrixC[i][j] = acc;	
            acc = 0;
        }
    }
\end{verbatim}

\subsection{Counters Used}
To measure the performance of our algorithm we gathered several counters from the system through \emph{PAPI}.
The following counters were used:
\begin{description}
\item[PAPI \_TOT\_CYC] Total number of cycles;
\item[PAPI \_TOT\_INS] Instructions completed;
\item[PAPI \_LD\_INS] number of load instructions;
\item[PAPI \_SR\_INS] number of store instructions;
\item[PAPI \_FP\_OPS] Floating point operations;
\item[PAPI \_FP\_INS] Floating point instructions;
\item[PAPI \_L1\_DCA] L1 data cache accesses;
\item[PAPI \_L1\_DCM] L1 data cache misses;
\item[PAPI \_L2\_DCA] L2 data cache accesses;
\item[PAPI \_L2\_DCM] L2 data cache misses;
\item[PAPI \_L3\_DCA] L3 data cache accesses;
\end{description}

\subsection{Tests}
\subsection{Results}
\subsubsection{Analysis Miss Rates}
\subsubsection{Analysis Memory Access}

\begin{thebibliography}{9}
\bibitem{roofline}
	\texttt{\small
	Roofline: An insightful Visual Performance Model for Floating-Point Programs and Multicore Architectures}	\\
	\emph{Samuel Webb Williams, Andre Waterman, David A. Patterson}	\\
	23th November 2012

\bibitem{ark}
	\texttt{\small
	http://ark.intel.com/products/35563/Intel-Core2-Duo-Processor-T9600-6M-Cache-2_80-GHz-1066-MHz-FSB}	\\
	\emph{Intel{\textregistered} Core {\texttrademark} 2 Duo Processor T9600 (6M Cache, 2.80 GHz, 1066 MHz FSB)}	\\
	{\copyright}Intel Corporation	\\
	23th November 2012

\bibitem{ark2}
	\texttt{\small
	http://ark.intel.com/products/43122/Intel-Core-i7-720QM-Processor-6M-Cache-1_60-GHz}	\\
	\emph{Intel{\textregistered} Core {\texttrademark} i7-720QM Processor(6M Cache, 1.60 GHz)}	\\
	{\copyright}Intel Corporation	\\
	23th November 2012

\bibitem{crucial}
	\texttt{\small
	http://www.crucial.com/store/ListParts.aspx?model=MacBook\%20Pro\%20\%28Early\%202008\%20and\%20Late\%202008\%29}	\\
	\emph{Computer memory upgrades for Apple MacBook Pro (Early 2008 and Late 2008) Laptop/Notebook from Crucial.com }	\\
	{\copyright} Micron Technology	\\
	24th November 2012

\bibitem{crucial2}
	\texttt{\small
	http://www.crucial.com/store/listparts.aspx?model=Pavilion\%20dv6-2190ep&Cat=RAM}	\\
	\emph{Computer memory upgrades for HP - Compaq Pavilion dv6-2190ep Laptop/Notebook from Crucial.com}	\\
	{\copyright} Micron Technology	\\
	24th November 2012	

\end{thebibliography}

\end{document}


