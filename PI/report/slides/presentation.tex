 \documentclass{beamer}

\usepackage{ucs}
\usepackage[utf8x]{inputenc}
\usepackage[T1]{fontenc}
\usepackage[english]{babel}
\usepackage[retainorgcmds]{IEEEtrantools}%	IEEEeqnarray
\usepackage{mathabx}%	convolution symbol
\usepackage{multi row}
\usepackage{epstopdf}
\usepackage{listings}
\lstset{
	language=c,
	basicstyle=\footnotesize,
	showtabs=true,
	tabsize=3,
}

%	presentation info
\title{Study and Optimization of a Finite Volume's Method Application}

\author{José Alves, Rui Brito}

\institute[pg22765, pg22781]{
	Universidade do Minho
}

\date{Braga, November 2012}


%	beamer options
\usetheme{Frankfurt}


\begin{document}%	begin presentation

\maketitle%	title slide

\begin{frame}
	\frametitle{Index}
	\tableofcontents
\end{frame}

\section{Case Study}

\begin{frame}[plain]
	\frametitle{Convexion-Diffusion}
	\begin{description}
		\item [What?]Simulates the way heat is transfered in a fluid;
		\item [How?] Uses finite-volumes method;
		\item [Why?] Represents surface as a mesh, making each cell only dependent of its neighbours;
	\end{description}
\end{frame}

\begin{frame}[plain]
	\frametitle{Convexion-Diffusion}
	\begin{description}
		\item [makeFlux] Calculates the contribution from each edge <- VER ISTO;
		\item [makeResidual] Calculates the $\phi$ vector, ading the flux for each cell, from each contribution <- VER ISTO;
		\item [LUFactorize] Calculate a Gauss elimination <- VER ISTO;
	\end{description}
\end{frame}


\section{Test Methodology}
\begin{frame}[plain]
	\frametitle{Test Machines}
		\begin{center}

			\begin{table}[!htp]
			\resizebox{\textwidth}{!}{
			\begin{tabular}{|c|c|c|c|}
			\hline
			 & compute-511-2@search & compute-611-1 & MacBookPro\\
			 & AMD Opt 6174 & Xeon X5650 & Intel Ivy-Bridge i7\\
			\hline
			\# processors & 2 & 2 & 1\\
			\# cores per processor & 12 & 6 & 4\\
			hyperthreading & - & yes & yes\\
			clock frequency(GHz) & 2.2 & 2.66 & 2.3\\
			L1 capacity & 128KB & 128KB & 64KB\\
			L2 capacity & 512KB & 256KB & 256KB\\
			L3 capacity & 12MB & 12MB & 6MB\\
			RAM capacity & 64GB & 48GB & 16GB\\
			\hline
			\end{tabular}}
			\caption{Test cases}
			\label{tab:testcases}
			\end{table}
		\end{center}
\end{frame}

\begin{frame}[plain]
	\frametitle{Test Parameters}
	\begin{enumerate}
		\item Best of 3 executions;
		\item Test for different number of threads;
		\item CENAS;
	\end{enumerate}
\end{frame}


\section{Sequential version}
\begin{frame}
	\frametitle{Original version}
	For each edge:
	\begin{enumerate}
		\item Calculate edge velocity;
		\item Calculate flux;
	\end{enumerate}
	For each cell:
	\begin {enumerate}
		\item Compute all contributions;
	\end{enumerate}
	Compute vector $\phi$;
	Compute a Gauss elimination;
	Compute the error;
\end{frame}

\begin{frame}
	\frametitle{Optimized version}
	\begin{enumerate}
		\item Reduce number of loads;
		\item Change some variable definitions to \emph{const};
		\item Usage of a recent compiler;
	\end{enumerate} 
\end{frame}


\section{Roofline}
\begin{frame}
	\begin{figure}[!htp]
		%\includegraphics[width=12cm]{images/roofline.eps}
		\label{fig:roofline}
	\end{figure}
\end{frame}

\section{PAPI Analysis}
\begin{frame}
	\frametitle{Counters Used}

	Used counters gathered by PAPI:
	\begin{description}
		\item[PAPI\_TOT\_CYC] Total cycles;
		\item[PAPI\_TOT\_INS] Total instructions
		\item[PAPI\_LD\_INS] Load Instructions
		\item[PAPI\_SR\_INS] Store Instructions
		\item[PAPI\_FML\_INS] Multiply instructions
		\item[PAPI\_FDV\_INS] Division instructions
		\item[PAPI\_VEC\_INS] Vector Instructions
		\item[PAPI\_FP\_OPS] Floating point operations
		\item[PAPI\_L1\_DCA] L1 data cache accesses
		\item[PAPI\_L1\_DCM] L1 data cache misses
		\item[PAPI\_L2\_DCA] L2 data cache accesses
		\item[PAPI\_L2\_DCM] L2 data cache misses
	\end{description}
\end{frame}

\begin{frame}
	\frametitle{PAPI comparison}
	\begin{figure}[!htp]
		%\includegraphics[width=12cm]{images/roofline.eps}
	\label{fig:original_papi}
	\end{figure}
	\begin{figure}[!htp]
		%\includegraphics[width=12cm]{images/roofline.eps}
	\label{fig:optm_papi}
	\end{figure}
\end{frame}


\section{Shared-Memory Parallel Optimization}
\begin{frame}
	\frametitle{OpenMP Objectives}
	\begin{enumerate}
		\item Parallelize application;
		\item Decrease runtime;
	\end{enumerate}
\end{frame}	


\begin{frame}
	\frametitle{Amdahl's Law}
		
		$S_{N}=\frac{1}{(1-P) + P/N}$

		\begin{center}
			\begin{table}[!htp]
			\begin{tabular}{lrlrlrl}
			\hline
			\textbf{Parallel Portion} & \textbf{\# Cores} & \textbf{Expected Speedup}\\
			\hline
			CENAS & 1 & CENAS \\
			CENAS & 2 & CENAS \\
			CENAS & 4 & CENAS \\
			CENAS & 8 & CENAS \\
			CENAS & 12 & CENAS \\
			CENAS & 16 & CENAS \\
			CENAS & 24 & CENAS \\
			\hline
			\end{tabular}
			\caption{Test cases}
			\label{tab:testcases}
			\end{table}
		\end{center}	
\end{frame}

\begin{frame}
	\frametitle{Achieved Results}
	\begin{figure}[!htp]
		%\includegraphics[width=12cm]{images/roofline.eps}
	\label{fig:optm_papi}
	\end{figure}
\end{frame}


%%%%%%%%%%%%%%%%%%%%%%%%%%%%%%%%%%%%%%%%%

\section{Conclusion}
\begin{frame}
	\frametitle{Conclusion}
	\begin{itemize}
		\item Some difficulties measuring memory ceilings;
		\item Lack of analysis of output from PAPI;
	\end{itemize}
\end{frame}

\section{Questions}
\begin{frame}
	\titlepage
	
	
\end{frame}

\end{document}%	end presentation